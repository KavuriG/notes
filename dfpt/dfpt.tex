\documentclass[]{article}
\usepackage{amsmath}
\usepackage{bm}

\title{DFPT: Density functional perturbation theory}
\author{}

\begin{document}

\maketitle



\section{The formalism of DFT}
        \subsection{Kohn, Sham and all that}

        We begin with the general many-body Schrodinger equation:

        \begin{equation} \label{eq:schrod}
          \bm{\widehat{H}}\Psi = \left( \sum_{i}^{N} \left(-\frac{\hbar^2}{2m_i}\nabla_i^2\right)
            + \sum_{i}^{N}V(\vec{r_i}) + \sum_{i<j}^{N}U(\vec{r_i}, \vec{r_j})\right)\Psi = E\Psi
        \end{equation}

        For a single electron, one defines the probability density to be given by
        $n(\vec{r}) = \Psi^*(\vec{r})\Psi(\vec{r})$ . In analogy, keeping in mind
        that electrons are indistinguishable, we can integrate over $N-1$ variables
        in the many-body function to arrive at the many body electron density:

        \begin{equation} \label{eq:dens}
          n(\vec{r}) = N\int d\vec{r_2}\int d\vec{r_3} \ldots \int d\vec{r_N}
          \Psi^*(\vec{r}, \vec{r_2}, \vec{r_3}, \ldots, \vec{r_N})  \Psi(\vec{r},
          \vec{r_2}, \vec{r_3}, \ldots, \vec{r_N})
        \end{equation}

        Further, we can arrive at a definition for a density operator that yields
        expectation value in \ref{eq:dens}. One can check that the following operator
        does the trick:

        \begin{equation} \label{eq:density}
                \widehat{n}(\vec{r}) = \sum_{i}^{N} \delta(\vec{r} - \vec{r_i})
        \end{equation}

        Now, we can write the expectation of energy as, from the Hohenberg-Kohn theorem:

        \begin{equation} \label{eq:funcham}
                E[n] = F[n] + \int V(\vec{r}) n(\vec{r}) dr^3
        \end{equation}

        Where the $F[n]$ is a \textit{universal} functional of density.

        To further gain an intuition about the nature of $F[n]$, we can defer to
        the \textit{constrained search} formalism of DFT, which gives the $F[n]$ as
        \begin{equation}\label{eq:constrain}
          F[n] = \min_{\{\Psi \mid \langle\widehat{n}(r)\rangle = n(r)\}}
          \langle \Psi| \widehat{T} + \widehat{H}_{e-e} |\Psi\rangle
        \end{equation}
        Even otherwise, te $F[n]$ has a similar interpretation.
        The key takeaway, though, is that the exact form of $F[n]$ is difficult to obtain
        explicitly as a function of $n$. Even though the Hohenberg-Kohn theorems make
        such a representation possible in theory,
        \begin{itemize}
        \item A given density $n$ fixes the $V(\vec{r})$, by HK.
        \item This fixes all terms in the many-body Schr{\"o}dinger equation, which
          is solved to give $\Psi$
        \item The above expectation value [\ref{eq:constrain}] is then taken.
        \end{itemize}

        there is no prescription for the first step, nor a general or easy way to
        deal with the second.

        One way to circumvent this difficulty is to write the unknown
        $\langle \widehat{H}_{e-e} \rangle$ at least partly as a funcional of density. We
        expect a large contribution to this to come from a simple couloumbic-Hartree
        like term $\int \int \frac{e^2 n(\vec{r}) n({\vec{r}\,}') }{2 |\vec{r}
          - {\vec{r}\,}'|} d\vec{r} d{\vec{r}\,}'$. All other effects are clubbed in under the $E_{xc}[n]$
        or the exchange-correlation energy.

        Now, the real crowning glory of DFT is what can be done using the
        \textit{Kohn-Sham ansatz}, which states that the correct ground state
        density of the full interacting system is \textit{exactly} the same as the
        ground state density of some unknown, non-interacting system. Based on this
        (details have been skipped), we can reduce the solution of the difficult problem
        of varying over the $ n(\vec{r}) $ into solving the self consistent Schr{\"o}dinger
        equation:
        \begin{equation}\label{eq:dft_schrod}
                [-\frac{\hbar^2}{2m}\nabla^2 + V_{hartree} + V_{xc} + V_{ext}]\Psi = \epsilon_o \Psi
        \end{equation}
        Here, the term $ V_{har.} $ acts as an effective electrostatic potential felt
        by an electron and corresponds to a similar term in the Hartree-Fock treatment.
        Explicitly, it is given by:

        \begin{equation} \label{eq:vhartree}
                V_{har.}({\vec{r}\,}') = \int dr' \frac{e^2 n({\vec{r}\,}')}{|\vec{r} - {\vec{r}\,}'|}
        \end{equation}

        All of the difficulty has thus been shoved under the carpet in the form of the
        term $V_{xc}$. This includes, something like the Hartree exchange effects and other
        unknown effects that arise from the fact that we haven't assumed a slater determinant
        ground state. We must next deal with this.
        Using adiabatic connection (details omitted), we get:
        \begin{equation}\label{eq:adiabat_cont}
        \begin{aligned}
          E(1) - E(0) &= \int_{0}^{1} \langle \Psi(\lambda)| \widehat{H}_{e-e}
          |\Psi(\lambda) \rangle d\lambda + \int_{0}^{1} d\lambda \int
          \frac{\partial V(r,\lambda)}{\partial \lambda}\langle\hat{n}(r)\rangle dr\\
          &= \int^{1}_{0} d\lambda \biggl\langle \sum_{i<j}^{N}U(\hat{r_i}, \hat{r_j})
          \biggr\rangle + \int [V(\vec{r}, 1) - V(\vec{r},0)]n(\vec{r})d\vec{r}\\
          &= \int^{1}_{0} d \lambda \int \int d \vec{r} d{\vec{r}\,}'
          \frac{e^2}{2|\vec{r} - {\vec{r}\,}'|} \frac{\biggl\langle \sum_{i,j}^{N}
           \delta(\hat{r_i} - \vec{r}) \delta(\hat{r_j} - {\vec{r}\,}') \biggr\rangle}
         {n(\vec{r})n({\vec{r}\,}')} + \mathrm{2^{nd}\  term}\\
         &= \int^{1}_{0} d \lambda \int \int d \vec{r} d{\vec{r}\,}'
         \frac{e^2}{2|\vec{r} - {\vec{r}\,}'|}\ g_\lambda(\vec{r}, {\vec{r}\,}')
         + \mathrm{2^{nd}\  term}\\
        \end{aligned}
      \end{equation}
      Where we have introduced the correlation function $g_\lambda(\vec{r}, {\vec{r}\,}')$. After a few
      manipulations, we arrive at the result:
      \begin{equation}
        \begin{aligned}
        \label{eq:9}
        E_{xc}[n] &= \int^{1}_{0}d\lambda \int \int d\vec{r} d{\vec{r}\,}' n({\vec{r}\,}')
        \frac{e^2}{|\vec{r} - {\vec{r}\,}'|} [g_\lambda(\vec{r}, {\vec{r}\,}') - 1]\\
        &= \int d\vec{r}\, n(\vec{r}) \int d{\vec{r}\,}'\, \frac{n_{xc}(\vec{r}, {\vec{r}\,}')}
        {|\vec{r} - {\vec{r}\,}'|}
        \end{aligned}
      \end{equation}
      We can also define and additional quantity representing the exchange-correlation
      energy per particle $\epsilon_{xc} = \int d{\vec{r}\,}'\, \frac{n_{xc}
        (\vec{r}, {\vec{r}\,}')}{|\vec{r} - {\vec{r}\,}'|}$. Physically, the value $n_{xc}(\vec{r}, {\vec{r}\,}')$ represents
      the density of an \textit{electron-hole} averaged over $\lambda$, and is the opposite sign as $n(\vec{r})$.
      



\end{document}
